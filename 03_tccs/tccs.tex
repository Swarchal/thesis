\documentclass[a4paper,11pt,twoside,openright]{scrbook}

\usepackage{swThesis}
\usepackage{amsmath}
\usepackage{standalone}
\usepackage{xcolor}
\standalonetrue

\bibliography{bibliography}

% Figures
\graphicspath{./figs}

\begin{document}

% basically the TCCS book chapter, but with more of an introduction

% compare other methods, PCA, cosine distance etc.

%\printbibliography

\chapter{Measuring distinct phenotypic response} \label{chapter:tccs}

\scriptsize{Note: this chapter is based on the previously published work: \textit{"Development of the Theta Comparative Cell Scoring Method to Quantify Diverse Phenotypic Responses Between Distinct Cell Types", S Warchal, J Dawson, N.O Carragher. ASSAY and Drug Development Technologies, pages 395-406, 7:14, 2016.}}



\section{Introduction}
\normalsize
\subsection{Comparing response to small molecules across a panel of cell lines}
% First introduce why you would want to compare phenotypic response between cell
% lines
Comparative analysis of panels of cell lines treated with compounds are routinely used in pharmacogenomic studies and drug sensitivity profiling, often using large numbers of cell lines and simple measures of compound response such as growth inhibition or cell death.
Studies such as these allow researchers to interrogate sensitivity of various small molecule therapies in a number of genomic backgrounds representing different diseases, disease-subtypes or patient populations.

Leveraging high-content screening methods in cell line panel studies allows the use of more complex cellular phenotypes than cell death, resulting in a more detailed characterisation of compound effect.
However, in order to apply multiparametric high-content data to pharmacogenomic studies, there needs to be a robust -- and ideally univariate -- measure of compound response to correlate drug sensitivity with genomic or proteomic datasets.


\subsection{Quantifying compound response in high content screens}
A simple but effective method to quantify the magnitude of compound response from multiparametric data is to calculate the distance from the negative control to the compound induced phenotype in feature space.
This idea was demonstrated by Tanaka \textit{et al.} who used PCA to reduce the dimensionality of the dataset to 3 principal components, and took the distance from the centroid of the negative control replicates to the compound co-ordinates in 3D principal component space. \cite{Tanaka2005}
Distance from the negative control in PCA space is an effective method for detecting phenotypically active compounds.
In addition, distance measurements can repeated for multiple concentrations of a compound to produce a concentration -- phenotypic-distance response curve (see figure \ref{figure:pca_dist}), and derived EC$_{50}$ values.
However, one issue of the distance-from-negative-control metric of compound activity is that it disregards much of the information relating to the position in feature space, as depicted in figure \ref{figure:pca_dist}, two compounds might have nearly identical distances, yet those distances may be produced by very different morphological changes.
For this reason I investigated incoporating directionality into the well-established PCA workflow.

\begin{figure}
    \captionsetup{width=0.8\textwidth}
    \caption[Compound distance in principal component space]{
    Diagram illustrating measuring magnitude of compound response by distance from the negative control centroid in principal component space.
    \textbf{(A)} Phenotypic distance to three different compounds.
    Compound A and B show phenotypic activity as they are distanced from the negative control cluster, whereas compound B shows little activity.
    Note that compound A and compound C have similar distances from the negative control centroid, yet have very different values in principal component space.
    \textbf{(B)} A titration series for each of the three compounds, depicting increasing concentration of compound increasing distance from the negative control as morphological changes become more apparent.
}
    \input{figs/ch3PcaDist.pdf_tex}
    \label{figure:pca_dist}
\end{figure}

% TODO: more info on principal components, eigen vectors?


\section{Phenotypic direction}
Using the existing image dataset of 24 compounds screened across 8 cell-lines at 8 semi-log concentrations (see chapter \ref{chapter:moa}, table \ref{table:compounds}) TODO ...


\section{Section 2}

\section{Discussion}




\section{Methods}

This chapter uses the same imaging dataset as in chapter \ref{chapter:moa}.

\subsection{Cells}
See chapter \ref{chapter:moa}.

\subsection{Compounds}
See chapter \ref{chapter:moa}.

\subsection{Staining}
See chapter \ref{chapter:moa}.

\subsection{Imaging}
See chapter \ref{chapter:moa}.

\subsection{Image analysis}
Images were analysed using Cellprofiler v2.1.1 to extract 309 morphological features.
Breifly, cell nuclei were segmented in the Hoechst stained image based on intensity, and clumped nuclei were separated based on shape.
Nuclei objects were used as seeds to detect cell-bodies in the cytoplasmic stains of the additional channels.
Subcellular structures such as nucleoli and Golgi apparatus were segmented and assigned to parent objects (cells).
From these objects morphological features were quantified.


\subsection{Data analysis}

\subsubsection{Data preprocessing}
Out of focus and low-quality images were detected through saturation and focus measurements and removed from the dataset.
Image averages of single object (cell) measurements were aggregated by taking the median of each measured feature per image.
Features were standardised on a plate-by-plate basis by dividing each feature by the median DMSO response for that feature and scaled by a z-score ($z$) to a zero mean and unit variance with the equation
\begin{equation} \label{eq:zscore}
    z = \frac{x - \mu}{\sigma}
\end{equation}
where $\mu$ is the mean and $\sigma$ is the standard deviation.

Feature selection was performed by calculating pair-wise correlations of features and removing one of a pair of features that have correlation greater than 0.9, and removing features with very low or zero variance.

% TODO:

The number of principal components

\subsubsection{subsubsection here}


\end{document}







%Random text before.
%
%text text text here:
%
%\begin{minted}[mathescape,
%linenos,
%numbersep=5pt,
%fontseries=Roboto Mono Medium,
%fontsize=\footnotesize,
%gobble=4,
%frame=leftline,
%framesep=2mm]{r}
%    # example code highlighting
%    dist <- function(x) {
%        return(sqrt(x %*% x))
%    }
%\end{minted}
%
%Example text inbetween.
%
%\begin{footnotesize}
%\begin{minted}[mathescape,
%linenos,
%numbersep=5pt,
%fontseries=Roboto Mono Medium,
%fontsize=\footnotesize,
%gobble=4,
%frame=leftline,
%framesep=2mm]{python}
%
%    # some comment
%    def new_function(x):
%        """docstring"""
%        total = 0
%        for i in x:
%            total += 1
%        return total
%
%\end{minted}
%\end{footnotesize}
