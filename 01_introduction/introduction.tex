\documentclass[a4paper,11pt,twoside,openright]{scrbook}

\usepackage{swThesis}
\usepackage{lipsum}
\usepackage{standalone}
\standalonetrue

% Bibliography
\bibliography{bibliography}

% Figures
\graphicspath{./figs}

\begin{document}

\chapter{Introduction} \label{chapter:intro}

\section{Eroom's Law: The increasing cost of drug discovery}
Throughout the last 70 years the cost of developing a new drug has steadily increased.
Scannel \textit{et al.} observed that the cost to develop a new drug has approximately doubled every 9 years \cite{Scannell2012}.
This observation has been dubbed ``Eroom's law'', a homage to Moore's law -- a well-known observation that the number of transistors in microprocessors approximately doubles every 2 years.
The cost of bringing a new drug to market is now approaching £1 billion, taking 10 years from initial concept to approval, the reasons behind this every-increasing cost are multi-faceted.
One explanation may be that the low-hanging fruit has been taken, effective long-standing remedies have been studied and commercialised, natural products screened, and we are now tackling the more complex diseases and pharmacological targets.
% pharma's productivity cliff
% lots of interest in diverse approaches to drug-discovery
% incorporate more technology intro drug-discovery

\section{The Drug Discovery Process}
\texttt{TODO:} A brief section on the drug discovery process, hit -> lead -> validation -> clinical trials -> approval.

\subsection{Target-Based Screening}
Over the past 30 years the majority of drug discovery programmes seized upon new advances in robotics and automation to screen ever-expanding compound libraries against hypothesised protein targets.
It would be difficult to argue that this target-based high-throughput screening (HTS) approach has not been fruitful, and has yielded many succesful therapeutics.
Despite HTS's clinical and commerical success stories, it is not a panacea, with a high attrition rate of candidate compounds once they enter clinical trials.
A large proportion of clinical trial failures are due to a lack of efficacy (\texttt{TODO:} get value), which can usually be traced back to an incomplete understanding of disease aetiology.

% issues with prior identification of a target
% large gamble on the hypothesised target
% examples where this has failed - Alzheimers

\subsection{Phenotypic Screening}
% define what is is
% how is this different to target-based
% historic approach
% examples of approved drugs through phenotypic screening (monastrol?)
% resurgence in recent years -> swinney and anthony, first in class
% pros, cons (complimentary to target-based, not a replacement)

\section{High content imaging}
% how this relates to phenotypic screening
% how this can be used in drug-discovery
% benefits over transcriptomics, proteomics etc
% downsides compared to alternatives

\section{Cancer}
% general background on breast cancer
% rationale behind picking the 8 cell lines
% cancer drug discovery
% applications of phenotypic/high-content imaging in cancer

%\printbibliography

\end{document}
