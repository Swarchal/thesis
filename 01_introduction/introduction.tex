\documentclass[a4paper,11pt,twoside,openright]{scrbook}

\usepackage{swThesis}
\usepackage{lipsum}
\usepackage{standalone}
\usepackage{mhchem}
\standalonetrue

% Bibliography
\bibliography{bibliography}

% Figures
\graphicspath{./figs}

\begin{document}

\chapter{Introduction} \label{chapter:intro}

\section{Eroom's Law: The increasing cost of drug discovery}
Throughout the last 70 years the cost of developing a new drug has steadily increased.
Scannel \textit{et al.} observed that the cost to develop a new drug has approximately doubled every 9 years \cite{Scannell2012}.
This observation has been dubbed ``Eroom's law", a homage to Moore's law -- an observation that the number of transistors in microprocessors approximately doubles every 2 years.
The cost of bringing a new drug to market is now approaching £1 billion, taking 10 years from initial concept to approval, the reasons behind this every-increasing cost are multi-faceted.
One explanation may be that the low-hanging fruit has been taken, effective long-standing remedies have been studied and commercialised, natural products screened, and we are now tackling the more complex diseases and pharmacological targets.


\section{Phenotypic Screening}
\begin{itemize}
\item define phenotypic screening
\item how it differs from target-based screening
\item why there is a resurgence in phenotypic drug discovery
\item benefits, downfalls
\end{itemize}

\subsection{High Content Imaging}
\begin{itemize}
\item define, multiparametric etc.
\item historic success stories
\item high-content profiling
\end{itemize}
%\printbibliography

\end{document}
