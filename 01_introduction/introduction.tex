\documentclass[a4paper,11pt,twoside,openright]{scrbook}

\usepackage{swThesis}
\usepackage{lipsum}
\usepackage{standalone}
\usepackage{caption}
\standalonetrue

% Bibliography
\bibliography{bibliography}

% Figures
\graphicspath{./figs}

\begin{document}

\chapter{Introduction} \label{chapter:intro}


\section{Eroom's Law: The increasing cost of drug discovery}
Throughout the last 70 years the cost of developing a new drug has steadily increased.
A study by Scannel \textit{et al.} noted the cost to develop a new drug has approximately doubled every 9 years \cite{Scannell2012}, this observation has been dubbed ``Eroom's law'', a homage to Moore's law -- the well-known observation that the number of transistors in microprocessors approximately doubles every 2 years.
The cost of bringing a new drug to market is now approaching £1 billion, taking 10 years from initial concept to approval, the reasons behind this every-increasing cost are still very much in debate, though most agree the issue is multi-faceted.
One explanation may be that the low-hanging fruit has been taken, effective traditional remedies have been studied and their active ingredients commercialised, natural products screened, leaving us to tackle the more complex diseases and pharmacological targets.  
This pessimism has led to the ever present idea that drug discovery is undergoing a productivity crisis, and that the investments made in early stage research do not translate into actionable pharmacology which can be used to develop efficacious therapies and ultimately benefit patients.
This belief has led to a renewed interest in alternative drug discovery paradigms.

% perceived decrease in productivity, leading to increased interest in other
% drug discovery paradiagms.
% lots of interest in diverse approaches to drug-discovery
% incorporate more technology intro drug-discovery

\section{The drug discovery process}

% Typically, commercial drug discovery programmes follow a well-established sequence of events, starting with selecting a disease or indication
% \begin{enumerate}
%     \item For a given disease or indication, an assay is developed which allows for high-throughput screening of large chemical libraries.
%     \item Hits are selected based on a univariate read-out. These are then confirmed with further repeats.
%     \item Lead compounds are then validated by testing in more complex assays, such as \textit{in vitro} or \textit{in vivo} disease models.
%     \item Validated compounds are taken through DMPK and toxicity testing.
%     \item A lead compound is selected and taken onto clinical trials.
%         \begin{enumerate}
%             \item phase I: Toxicity and DMPK testing.
%             \item phase II: Effectiveness at treating the condition.
%             \item phase III: Larger scale testing of efficacy.
%         \end{enumerate}
% \end{enumerate}
% \texttt{why is this being written about??}


\subsection{Target-based screening}
Over the past 30 years the majority of drug discovery programmes have seized upon technological advances in robotics and automation to screen ever expansive compound libraries against pre-defined protein targets.
It would be difficult to argue that this target-based high-throughput screening (HTS) approach has not been fruitful, yielding many succesful therapeutics across a range of disease areas, largely attributed to an increased understanding of the genomic basis of many diseases.
However, despite numerous clinical and commerical success stories, HTS is not a panacea, with a high attrition rate of lead compounds once they enter clincial trials \cite{citation_needed}.
A large majority of these clinical trial failures are not due to toxicity, but rather a lack of efficacy which can often be traced back to a poorly hypothesised target in the face of complex disease aetiology.


% issues with prior identification of a target
% large gamble on the hypothesised target
% examples where this has failed - Alzheimers? neuroscience in general?

\subsection{Phenotypic screening}
Phenotypic screening differs from target-based screening in that it does not rely on prior knowledge of a specific target, but instead interrogates a biologically relevant assay to identify compounds which alter the phenotype in a biologically desirable way.
This target-agnostic approach can prove useful in diseases with poorly understood mechanisms, or those with no obvious druggable protein targets.
Phenotypic screening is not a new approach in small molecule drug discovery, it was the primary method for many decades before the genomics revolution made target hypothesis tractable
\cite{Zheng2013}.

An example of a modern day target-agnostic phenotypic screen is the development of the anti-hepatitis C drug daclatasvir.
A model was developed in which human cells were modified to express the hepatitis C virus (HCV) replicon, this was then screened to identify compounds which reduced HCV replication.
The target for daclatasvir was later found to be the HCV NS5A protein, which is a novel target and likely not have been suggested in a hypothesis-driven screen as at the time it was thought to have no involvement with HVC. \cite{Belema2014,Nettles2014}.
This elegant assay has the very useful attribute of modelling the disease whilst simulatenously identifying compounds which may prove to be toxic in human cells.

Many concerns related to phenotypic screening are centered on the lack of mechanistic information for a given lead compound.
Whilst the lack of a known target may cause concerns within a commercial drug discovery programme, regulatory bodies such as the Food and Drug Administration (FDA) and European Medicines Agency (EMA) do not require a known target for drug approval -- only that it is safe and efficacious.
Metformin, a first-line therapy for type 2 diabetes, and is on the World Health Organisation's list of essential medicine, decreases liver glucose production and has a insulin senstitising effect on many tissues.
Despite approval in Europe since 1957 and widespread clinical use, the  molecular mechanism of metformin remained unknown for 43 years \cite{Hundal2000}.
Although knowledge of the molecular target is not necessary to get a drug into the clinic, target deconvolution is still an important part of phenotypic drug discovery programmes.
Without knowing which protein or proteins a compound is binding to, lead optimisation via structure activity relationship (SAR) studies becomes extremely difficult.
In addition, knowledge of the molecular target of a lead compound generated by a phenotypic screen can be used as a basis for starting a more high-throughput hypothesis-driven screen on a novel target.
It is for this reason that many view phenotypic screening as a complimentary method to target based screening, rather than a competing approach or a proposed replacement.



% resurgence in recent years -> swinney and anthony, first in class

\section{High content imaging}
High content imaging is a technique utilising high-throughput microscopes and automated image analysis, commonly used in phenotypic screening as a method for gathering multivariate datasets from images of biological specimens and has proven useful in a wide variety of phenotypic assays, ranging from 2D mammalian cells \cite{cite_HCA_cell_papers}, \textit{in vivo} studies in zebrafish \cite{GeoffreyBurns2005} and even plants and crops \cite{Chen2014}.

High content screens -- screening studies carried out with high content imaging -- are particularly useful in phenotypic drug discovery for a few reasons.
The first is that they allow the use of more complicated assays, which might better represent the biological complexity than simple reductionist models.
However, these complex assays often have readouts which are more difficult to quantify, which a single univariate readout may fail to accurately recapitulate,
therefore the multivariate datasets produced by high content screening enables a more detailed representation of a complex assay endpoint.
A second benefit is that the multivariate nature of the data generated by high content screening facilitates a more unbiased method for selecting hits.
% TODO: expand on this point?

% how this relates to phenotypic screening
% how this can be used in drug-discovery
% benefits over transcriptomics, proteomics etc
% downsides compared to alternatives
% high-dimensional profiling of compound response


\section{Cancer drug discovery}
Cancer drug discovery programmes of past decades seized upon uncontrolled proliferation as a clinically relevant phenotype to screen against, giving rise a number of cytotoxic anti-proliferative and cytotoxic compounds, often renowned for their severe side-effects.
Many modern day oncology drug discovery programmes still retain anti-proliferation as a key marker for pre-clinical success, although our ever increasing knowledge of cancer's molecular underpinnings has driven many oncology programmes towards a more target-based approach.
The prototypical success story of target-driven drug discovery in oncology is imatinib, a tyrosine kinase inhibitor targetting the bcr-abl fusion protein in chronic myeloid leukemia.
However, despite imatinib's success, in most cases targeting a single driver in a complex signalling network results in compensatory signalling, activation of redundant pathways, and unpredicted feedback mechanisms which often diminish or completely amilerorate efficacy \textit{in vivo}.

% rationale behind picking the 8 cell lines
% applications of phenotypic/high-content imaging in cancer
% use and cite the moffat paper here


\begin{table}[]
    \begin{footnotesize}
    \captionsetup{width=0.60\textwidth}
    \centering
    \caption[Panel of breast cancer cell lines chosen for study]{Panel of breast cancer cell lines chosen for study. PI3K:Phosphoinsitide-3-kinase, PTEN:Phosphatase and tensin homolog, ER:Estrogen receptor, TN:triple-negative, HER2:human epidermal growth factor, WT:wild-type, $\star$:lack of consensus regarding the mutational status.}
    \label{table:cell-lines}
    \begin{tabular}{@{}llll@{}}
    \toprule
               &                    & \multicolumn{2}{l}{Mutational status} \\
    Cell line  & Molecular subclass & PTEN             & PI3K               \\ \midrule
    MCF7       & ER                 & WT               & E545K              \\
    T47D       & ER                 & WT               & H1047R             \\
    MDA-MB-231 & TN                 & WT               & WT                 \\
    MDA-MB-157 & TN                 & WT               & WT                 \\
    HCC1569    & HER2               & WT               & WT                 \\
    SKBR3      & HER2               & WT               & WT                 \\
    HCC1954    & HER2               & $\star$          & H1047R             \\
    KPL4       & HER2               & $\star$          & H1047R             \\ \bottomrule
    \end{tabular}
    \end{footnotesize}
\end{table}


%\printbibliography

\end{document}
