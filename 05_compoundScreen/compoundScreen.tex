\documentclass[a4paper,11pt,twoside,openright]{scrbook}

\usepackage{swThesis}
\usepackage{amsmath}
\usepackage{lipsum}
\usepackage{standalone}
\standalonetrue

\bibliography{bibliography}

% Figures
\graphicspath{./figs}

\begin{document}

\chapter{Large compound screen across 8 breast cancer cell lines} \label{chapter:screen}

% description of screen(s)
% proteomic and transcriptomic work to identify pathways responsible for the
% differential response

% small molecule inhibition of highlighted pathways to validate their
% involvement

% function assays to investigate biological function of observed phenotypes
    % cell-cycle
    % apoptosis
    % spheroids

\section{Introduction}

\subsection{subsection}



\section{Results}

\subsection{Hit selection}

\subsection{Serotonin related compounds}

\subsection{Spheroids}

\subsection{RPPA}


\section{Methods}

\subsection{Identifying hits}

\subsection{Spheroids}

\subsubsection{Creating spheroids}
Spheroids were created  by seeding approxmiately 10,000 GFP-expressing cells per well in 50 $\mu$L of media into each well of a 96-well low attachment U-bottomed plate (\#7007 Corning).
A solution containing 4\% growth-factor reduced Matrigel (\#35623 Corning) and 2\% DRAQ7 apoptotic stain (\#DR710HC biostatus) was made in cold media, and 50 $\mu$L per well was added to the existing cell suspension, for a final Matrigel concentraion of 2\% and 1\% DRAQ7.
Plates were then centrifuged for 10 minutes at 1000X G and 4$^\circ$ with break speed reduced to pellet down the cells in the centre of each well.
After centrifugation plates were placed in a tissue culture incubator for 24 hours before addition of compounds.
Compounds from a 1000x source plate were diluted 1:50 by transferring 3 $\mu$L from the source plate to an intermediate plate containing 150 $\mu$L of media.
From the intermediate plate 5 $\mu$L were transferred to the spheroid assay plate containing 100 $\mu$L for a final dilution of 1:1000 and a DMSO concentration of 0.1\%.
Following compound addition spheroid plates were incubated for an additional 72 hours.

\subsubsection{Imaging spheroids}
Spheroids were imaged on the ImageXpress using the 4X objective lens in 3 channels (transmitted light, GFP and CY5).
Images were captured by first detecting the well-bottom in the centre of the U-bottomed well with a laser-based autofocus and offsetting by the well thickness, then capturing images in a z-stack at 8 focal planes spaced at 50 $\mu$m intervals for a total range of 350 $\mu$m.
Z-stacks of the GFP and CY5 fluorescent channels were collapsed into a single image per channel using a maximum intensity projection, while the z-stack of transmitted light images were transformed using a minimum intensity projection.




\subsection{Western blotting for SERT and TPH1}

\subsubsection{Protein extraction}

\paragraph{Protein extraction from cells.}
Cells were grown for protein extraction in 6-well plates and lysed when roughly 80\% confluent.
RIPA buffer was made from 25 mM Tris (pH 7.5), 150 mM NaCl, 0.1\% SDS, 1\% Triton-X100, 0.5\% deoxycholate and phosphatase inhibitors (cOmplete ULTRA \#05892970001 Roche), and placed on ice.
Media was removed from the 6-well plates with a pipette, and wells were washed with cold PBS before addition of 300 $\mu$L of cold RIPA lysis solution.
Cells were incubated in lysis buffer for 2-3 minutes on ice and scraped with a cell-scraper. The lysis buffer/cell-lysate was then transferred to a 1.5 mL eppendorf tube and incubated on ice for 10 minutes.
Cell lysate solutions were centrifuged at 13,000x G for 10 minutes at 4 $^\circ$C and the supernatent transferred to new 1.5 mL eppendorf tubes.


\paragraph{Protein extraction from mouse brain.}
Two grain of rice sized pieces were cut from different regions of a frozen FVB adult mouse brain and separately homogenised (MP lysing matrix type D 1.4mm sphere) for 20 seconds at 4 M/s (MP FastPrep-24).
Once homogenised, 500 $\mu$L of RIPA lysis buffer was added to each tube and placed on a rotator for 15 minutes at 4$^\circ$C.
The homogenising tubes and lysis solution were then briefly centrifuged to settle the beads and the lysates extracted with a pipette into pre-chilled 1.5 mL eppendorf tubes which were centrifuged at 13,000x G for 10 minutes at 4$^\circ$C and the supernatents were transferred to new 1.5 mL tubes.


\paragraph{Standardising protein concentrations.}
Protein concentrations were measured using Precision-Red protein assay reagent (\#ADV02-A Cytoskeleton, Inc.).
For each sample, 10 $\mu$L of lysate was added to a plastic cuvette followed by 1 mL of Precision-Red.
The absorbance was measured after 1 minute incubation at room temperature with a spectrometer at 500 nm and compared to a blank of 1 mL Precision-Red and no lysate.
Protein concentration in mg/mL was determined as 100 times the corrected absorbance value at 600 nm, as per the suppliers instructions.
Sample protein concentrations were all made to 2.5 mg/mL by diluting in RIPA lysis buffer.
Samples were then prepared for Western blot by mixing 75 $\mu$L of sample with 25 $\mu$L of 4x loading buffer (200 mM Tris, 400 mM DTT, 8\% SDS, 0.4\% bromophenol blue, 40\% glycerol) denaturing at 90$^\circ$C for 10 minutes.



\subsubsection{Western blotting}

\paragraph{Gel eletrophoresis and transfer.}
Using pre-cast 4-15\% gradient gels (\#4568085 Bio-Rad) and SDS-page running buffer (20 mM Tris, 195 mM Glycine, 0.1\% SDS, pH 8.5), 7.5 $\mu$L of seeBlue Plus2 ladder (\#LC5925 Invitrogen) and 20 $\mu$L of sample (50 $\mu$g) were added to separate lanes, and the gel was run at 200V constant voltage for 30 minutes.
Proteins were transferred to PVDF (polyvinylidene difluoride) membranes (P0.45 Aversham hybond) pre-wetted in in MeOH sandwiched with blotting paper at 400 mA constant ampage for 60 minutes in transfer buffer (SDS-page running buffer with 20\% MeOH) with ice packs.



\paragraph{Antibody staining and visualising.}
Following transfer, membranes were blocked in 20 mL of 1\% casein blocking solution (\#11921681001 Roche) for 10 minutes on a rocking plate.
Antibody solutions were made up in the same blocking buffer, TPH1 (\#LS-C117936 LSBio) was diluted 1:500, SERT (\#AMT-004 Amalone) was diluted 1:500.
Membranes were placed in separate 50 mL Falcon tubes with the protein side facing inwards in 7 mL of antibody solution and placed on rollers overnight at 4$^\circ$C.
The membranes were then removed from the antibody solution was washed 3 times with 10 mL of TBS-T (50 mM Tris, 150 mM NaCl, 0.1\% Tween 20, pH 7.5) for 5 minutes.
The secondary antibody (HRP-linked anti-rabbit IGg \#70742 CST) was diluted 1:5000 in blocking buffer and used to wash the membranes for 1 hour at room temperature.
Membranes were then washed 3 times with 10 mL TBS-T for 5 minutes followed by 3 washes with 10 mL TBS (50 mM Tris, 150 mM NaCl, pH 7.5) for 5 minutes.
Imaging of the membrane was carried out using BM chemiluminescence blotting substrate (\#11500619001 Roche) following the manufacturers instructions.

TODO:housekeeper control, imaging


\subsection{RPPA}

\subsubsection{Protein extraction}

\paragraph{2D cells.}
Protein extraction from 2D cells was performed by first seeding approximately 50,000 cells per well of a 6-well plates in 3 mL of media followed by incubation in a tissue culture incubator for 24 hours.
Compound addition was performed by diluting compound stocks in DMSO 1:50 in media to an intermediate plate, followed by 1:20 from the intermediate plate to the assay plate for a 1000-fold dilution and 0.1\% DMSO.
Assay plates were then incubated for an additional 72 hours, after which wells were washed with 1 mL of room temperature PBS followed by addition of 100 $\mu$L of room temperature CLB1 (Zeptosens, Bayer) lysis buffer.
Cells and lysis buffer were then scraped into 1.5 mL eppendorf tube and incubated at room temperature for 30 minutes with frequent vortexing.
After 30 minutes of incubation lysis solution was centrifuged for 10 minutes at 13,000X G at room temperature and supernatant was transferred into new 1.5 mL eppendorf tubes.

\paragraph{Spheroids.}
Protein extraction from spheroids was performed by first growing spheroids in 96-well plates following the same protocol as for imaging.
20 spheroids per treatment group were extracted with a pipette into a 1.5 mL eppendorf tube.
Pipette tips were widened by cutting with scissors.
The spheroids were then centrifuged for 30 seconds at 13,000X G at room temperature to pellet at the bottom of the tube, media was removed with a pipette and replaced with room temperature PBS.
Spheroids were pelleted again, PBS removed and replaced with 75 $\mu$L of room temperature CLB1 lysis buffer.
The spheroid lysis buffer mixture was incubated at room temperature for 30 minutes with frequent vortexing to break up cell aggregates.
Following incubation the lysis solution was centrifuged for 10 minutes at 13,000X G at room temperature, and supernatant extracted into a new 1.5 mL eppendorf tube.

\paragraph{Determining protein concentration.}
Protein concentration was determined with a Bradford assay, using a standard curve of known BSA concentrations and the addition of CLB1 lysis buffer to control for the lysis buffer concentration of the samples.
A curve of known BSA concentrations was created using 2 mg/mL BSA protein standard (\#23209 Thermo Scientific) diluted in PBS, with a 1:20 concentration of lysis buffer (see table \ref{table:bradford}).
Samples were diluted 1:20 in PBS by adding 2.5 $\mu$L of sample to 47.5 $\mu$L of PBS and mixed with a vortex.
10 $\mu$L of diluted samples and standard were added to each well of a flat-bottomed 96-well plate, followed by 240 $\mu$L of room temperature Coomassie Plus Protein Assay (\#1856210 Thermo Scientific) and incubated at room temperature for 10 minutes.
Plates were then read with a microplate reader (BIORAD iMark) at a wavelength of 595 nm.
The protein concentrations of samples were calculated from a linear model of the BSA standard curve.
All protein samples were normalised to 1 mg/mL by dilution in CLB1 lysis buffer.

% Please add the following required packages to your document preamble:
% \usepackage{booktabs}
\begin{table}[h]
    \begin{footnotesize}
    \captionsetup{width=0.8\linewidth}
    \centering
        \caption[Bradford standard BSA curve]{
            Volumes for the BSA standard curve. Lysis buffer was CLB1, the same as used for the sample preparation.
        }
    \label{table:bradford}
    \begin{tabular}{@{}llll@{}}
        \toprule
        \begin{tabular}[c]{@{}l@{}}BSA final\\ concentration (mg/mL)\end{tabular} & \begin{tabular}[c]{@{}l@{}}BSA 2 mg/mL\\ ($\mu$L)\end{tabular} & PBS ($\mu$L) & Lysis Buffer ($\mu$L) \\ \midrule
            0                                                                         & 0                                                              & 95           & 5                     \\
            0.05                                                                      & 2.5                                                            & 92.5         & 5                     \\
            0.1                                                                       & 5                                                              & 90           & 5                     \\
            0.15                                                                      & 7.5                                                            & 87.5         & 5                     \\
            0.2                                                                       & 10                                                             & 85           & 5                     \\
            0.3                                                                       & 15                                                             & 80           & 5                     \\
            0.4                                                                       & 20                                                             & 75           & 5                     \\
            0.6                                                                       & 30                                                             & 65           & 5                     \\ \bottomrule
    \end{tabular}
    \end{footnotesize}
\end{table}



%\printbibliography

\end{document}
