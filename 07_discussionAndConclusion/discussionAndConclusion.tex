\documentclass[a4paper,11pt,twoside,openright]{scrbook}

\usepackage{swThesis}
\usepackage{amsmath}
\usepackage{lipsum}
\usepackage{standalone}
\standalonetrue

\bibliography{bibliography}

% Figures
\graphicspath{./figs}

\begin{document}

\chapter{Concluding remarks} \label{chapter:discussionAndConclusion}

\section{Summary of completed work}

This work describes my research into the development and application of high-content image-based screening methods in 
the context of cancer drug discovery.
The first results chapter builds on published literature to further investigate how compound induced morphological 
changes measured with high-content imaging can be used to describe and inform the compound mechanism-of-action across a 
panel of genetically distinct breast cancer cell lines.
A comparison of two machine learning approaches revealed that, in fairly naive implementations, there is not a large 
difference in predictive performance between tree-based ensemble classifiers trained on extracted morphological 
measurements and CNN classifiers trained on pixel values.
There is however a difference in how well these two types of models can generalise to new data from new or unseen 
cell-lines, as the extracted morphological measurements used with the tree-based classifier can be more easily 
normalised which in turn affects how the addition of data during training from morphologically distinct cell-lines 
impacts model performance.
In chapter \ref{chapter:tccs} I described the development of a measure of morphological dissimilarity.
Inspired by a talk given by Simon Gordonov\footnote{A talk at an SBI$^2$ meeting describing phenotypic directions in 
principal components which has since been published\cite{Gordonov2016}}, I thought to extend the idea of measuring 
distance from a negative control in principal component space of morphological features to incorporate the idea and 
quantification of phenotypic direction.
This method was then applied in chapter \ref{chapter:screen} with a small molecule screen of approved drugs across the 
eight breast cancer cell-lines to identify compounds that produced distinct phenotypic responses between the cell-lines.
These compounds were then further investigated in 2D and 3D tumour spheroid assays of cell-viability, and proteomics 
performed the levels and activation state of common cancer cell growth and survival signalling pathways in cells 
treated with compounds grown in 2D and 3D environments.
The final results chapter is my effort to incorporate data from small molecule high-content screening studies with data 
relating to chemical structure.
The original premise behind this work was the hypothesis that structurally similar molecules are likely to produce 
similar morphological changes in cells.
I found evidence of correlation between chemical similarity and phenotypic similarity, although the effect size was 
extremely small.
Perhaps more interesting was the use of chemical structure data and existing chemical databases to generate hypotheses 
towards mechanistic understanding of hits found with target-agnostic screens, as well as high-content screens to 
identify compounds from interesting and rarely explored areas of chemical space.

During this work I spent a considerable amount of time developing software tools, either to implement new ideas or to 
streamline repetitive workflows which are commonplace in screening assays.
One of the biggest challenges was the time taken to analyse the millions of images generated by compound screens across 
a panel of cell-lines.
Whilst microscope vendors typically have software to automatically analyse and quantify images once acquired, in my 
case the software was limited by licences to a single desktop and unable to analyse the 12,000 compound screen in a 
reasonable amount of time.
Image analysis tools such as Cellprofiler, EBImage and HCS-analyser with permissive licences allow the analysis of 
thousands of images in parallel using distributed processing across compute clusters.
The most useful tool I developed was used to link the ImageXpress images to Cellprofiler running on the University's 
high-performance compute cluster\footnote{https://github.com/carragherlab/cptools2}, enabling the analysis of a 5 
million image dataset in roughly 24 hours, which would have otherwise taken months using the vendor supplied image 
analysis software.


\section{Remarks, unanswered questions and new questions}

High-content analysis and screening is an evolving field and has not yet reached a consensus on established workflows 
or best practices, with numerous labs developing their own image analysis software and data handling pipelines in 
isolation.
Recently there has been some effort to coordinate sharing methods and ideas between groups to converge on a 
standardised workflow. \footnote{http://cytodata.org/ \\
https://github.com/shntnu/cytomining-hackathon-wiki/wiki}
While this is an important step, high-content analysis has the enviable position of being at the crossroads of computer 
vision, multivariate analysis and machine learning, all of which are rapidly developing fields in their own right.
Therefore, despite efforts to reach an agreement on some form of standardised workflow, there is the conflicting 
temptation for researchers to adopt the latest tools and techniques in their analyses.

With the rapid development of new machine learning tools, particularly in computer vision, I envisage that the field 
will adopt these technologies where they show increased performance over hand-crafted algorithms in areas such as 
segmentation, \cite{Ronneberger2015,Sadanandan2018} feature extraction \cite{Yosinski2014,Razavian2014} and image 
classification. \cite{Russakovsky2015}
However, the use of ``classical'' extracted morphological features from images such as cell area or nuclei intensity 
offer a huge advantage in their simplicity, interpretability and the ability to investigate specific biological 
questions or image-analysis tasks.

With the increasing ability to generate large multivariate datasets from high-content screening, perhaps a more 
pertinent area of research is how best to leverage this data to improve our understanding of biological processes and 
find new and efficacious drugs for patients.
The interpretation of large multivariate datasets in biology is not unique to high-content imaging and is a task shared 
in common with most -omics technologies, with the only difference is that high-content imaging is usually cheaper than 
its -omics counterparts per sample -- and as a result sample sizes are typically much larger.
This commonality between technologies will hopefully lead to the development of new methods which are applicable to 
drug screening studies, which have historically relied upon univariate measures and statistical assumptions that do not 
necessarily hold true with more complex datasets.

In my opinion, the ``profiling'' of perturbations such as small molecule treatments or gene knock-outs, while certainly 
possible with high-content imaging, may benefit more from the standardised measured features such as L100h0 gene 
expression profiles of the connectivity map \cite{Lamb2006,Subramanian2017} which allow far easier comparisons and 
meta-analyses of disparate datasets in lieu of increased costs and lower throughput.
However, the low-cost and high-throughput of high-content screening is ideally suited for drug discovery projects using 
complex disease-relevant models which require multivariate measurements in order to accurately capture and quantify 
their complexity.

To conclude, I have presented work relating to a number of varied aspects of image informatics and high-content 
screening, these contributions are part of a rapidly developing field with many remaining questions and unverified 
assumptions.
As the field of biology progresses towards generating ever larger and more complex datasets there needs to be a similar 
progression in the research and development of data analysis methods to gain more understanding from the data we 
generate.
It is my hope that the evolution of new biological and analytical methods which enable in-depth profiling of compound 
mechanism-of-action and target biology across more complex \textit{in vitro} models of disease will better lead early 
stage drug discovery programmes towards increased clinical success rates.

\end{document}
